\section*{Summary of \textit{Semi-supervised classification with graph convolutional network}}
\subsection*{Introduction}
This paper introduces a scalable approach for semi-supervised learning on graph-structured data based on a variant of cnn that operate directly on graphs.\\
This particular model scales linearly in the number of edges and learns hidden layers representations that encode the local graph structure and feature of the nodes.\\
Classifying nodes in a graph where only a handful of nodes has access to labels can be looked at as a graph based semi-supervised learning problem where labels are smoothed via Laplacian regularization term in the loss function.\\
Below is the function:
\begin{equation*}
	\mathcal{L} = \mathcal{L}_0 + \lambda\mathcal{L}_{reg}
\end{equation*}
with:
\begin{itemize}
	\item \begin{equation*}\mathcal{L}_{reg} = \sum_{i,j} A_{i,j} \lVert f(X_i)-f(X_j) \rVert^2 = f(X)^T\Delta f(X)\end{equation*}
	\item $\mathcal{L}_0$ is the supervised loss.
	\item f can be a neural network-like differentiable function.
	\item $\lambda$ is a weighting factor.
	\item X represents the matrix of note feature vectors.
	\item A is an adjacency matrix.
	\item D is a degree matrix.
	\item $\Delta = D - A $ 
\end{itemize}
In this paper the graph structure is encoded as f(X, A) and trained on a supervised target $\mathcal{L}_0$ for all nodes who have access to labels. By doing so it is possible to aboid explicit regularization in the loss function.\\
The f function allows the model to distribute gradient information from $\mathcal{L}_0$ and enables it to learn representation of all the nodes.\\
The work can be divided into two parts:
\begin{enumerate}
	\item show that it is possible to use a first-oder approximation of spectral graph convolutions.
	\item demonstrate that this kind of graph based model can be used for semi-supervised classification of graph nodes.
\end{enumerate}
\subsection*{Spectral graph convolutions}
They consider spectral convolutions defined as:
